\documentclass[11pt]{article}

\title {Lesson 1 Exercise Solutions}
\author {Tyler Wang}

\usepackage{amssymb}
\usepackage{amsmath, esint}
\usepackage{amsthm}
\usepackage{array}
\usepackage{mathtools}
\usepackage{pdfpages}
\usepackage{fancyhdr}
\usepackage{lastpage}
\usepackage[figurename=Figure]{caption}
\usepackage{empheq}
\usepackage{mdframed}
\usepackage[a4paper, left=1in,right=1in,top=1in,bottom=0.7in,footskip=0.6in,includeheadfoot]{geometry}
\usepackage{tikz}
\usepackage{pgfplots}

\usetikzlibrary{calc,backgrounds,angles,quotes}

\tikzset{
    partial ellipse/.style args={#1:#2:#3}{
        insert path={+ (#1:#3) arc (#1:#2:#3)}
    }
}

\setlength{\parskip}{10pt}
\setlength{\parindent}{0pt}

\makeatletter
\edef\mytitle{\@title}
\edef\mydate{\@date}
\edef\myauthor{\@author}
\makeatother

\fancyhf{}
\fancyhead[R]{\myauthor}
\fancyhead[L]{\mytitle}
\fancyfoot[c]{Page \thepage \hspace{1pt} of \pageref{LastPage}}
\pagestyle{fancy}

\newmdtheoremenv{lemma}{Lemma}
\newmdtheoremenv{prop}{Proposition}
\numberwithin{lemma}{section}
\numberwithin{equation}{section}
\numberwithin{prop}{section}

\def\real{\mathbb{R}}
\def\complex{\mathbb{C}}
\def\rat{\mathbb{Q}}
\def\nat{\mathbb{N}}
\def\integ{\mathbb{Z}}
\def\mod#1{\mathbb{Z}_{#1}}
\def\cpx{\mathbb{C}}

\def\paren#1{\left(#1\right)}
\def\sbrak#1{\left[#1\right]}
\def\cbrak#1{\left\{#1\right\}}

\def\ceil#1{\left\lceil #1 \right\rceil}
\def\floor#1{\left\lfloor #1 \right\rfloor}
\def\abs#1{\left\lvert #1 \right\rvert}
\def\abrak#1{\left\langle #1 \right\rangle}
\def\bra#1{\left\langle #1 \right\rvert}
\def\ket#1{\left\lvert #1 \right\rangle}
\def\braket#1#2{\left\langle #1 \left.\right\lvert #2 \right\rangle}

\def\jand{\quad\text{and}\quad}
\def\jor{\quad\text{or}\quad}
\def\for{\quad\text{for }\,}

\def\ieval#1#2{\Bigg|^{#2}_{#1}}
\def\deval#1{\bigg|_{#1}}

\def\diff#1#2{\frac{d#1}{d#2}}
\def\pdiff#1#2{\frac{\partial#1}{\partial#2}}

\def\sec#1{\section*{#1}\addtocounter{section}{1}\setcounter{subsection}{0}}

\setlength\extrarowheight{3pt}

\begin{document}
\sec{Question 1}
\begin{enumerate}
	\item $A\cup B=\{1,\beta,20,\text{two}\}$
	\item $A\cup B=\{\beta,20\}$
	\item $C\setminus A=\{\text{one},\alpha\}$
	\item $\{1,\beta,20\}$
\end{enumerate}

\sec{Question 2}
\begin{enumerate}
	\item $\{x:-10<x<3\}$
	\item $\{2n:n\in\int\}$
	\item $\{n^2:n\in\nat\}$
\end{enumerate}

\sec{Question 3}
Checking the base case $n=1$ is trivial.
Then for the inductive step, assume for $m\in\nat$
$$\sum^m_{k=1}q^k=\frac{q^{m+1}-q}{q-1}$$
Then for $m+1$,
$$\sum^{m+1}_{k=1}q^k=\sum^m_{k=1}q^k+q^{m+1}
=\frac{q^{m+1}-q}{q-1}+q^{m+1}$$
$$=\frac{q^{m+1}-q}{q-1}+\frac{q^{m+2}-q^{m+1}}{q-1}=\frac{q^{m+2}-q}{q-1}$$
therefore, by induction, we complete our proof.
\sec{Question 4}
Checking the base case $n=1$ is trivial.
Then for the inductive step, assume for $m\in\nat$
$$5^m+5<5^{m+1}$$
This implies
$$5^{m+2}>5\cdot 5^{m+1}>5(5^m+5)>5^{m+1}+25>5^{m+1}+5$$
hence by induction, we complete our proof.

\sec{Question 5}
Let $T=\{2n+1:n\in\nat\}$, since $T\subseteq\nat$, $T$ is also well ordered. Then the proof follows by contradiction. 
Assume there exists a set $S=\{P(n):n\in T\}$ such that there exists $P(m)$ that is false for $m\in T$. 
Therefore, there exists $F=\{P(n):P(n)=\text{False}\}\neq \varnothing$. Since $S$ is well-ordered (since $T$ can be used to index $S$) and $F\subseteq S$, $F$ must have least element $P(l)$. Since $l\neq1$, by base case, there exists $l-2\in T$. Since $P(l-2)\notin F$, $P(l-2)$ is true, which implies $P(l)$ is true by our inductive step assumption. Therefore, by contradiction, every $P(n)\in S$ must be true.
\end{document}




